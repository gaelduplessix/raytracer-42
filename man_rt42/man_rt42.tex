\documentclass{report}

\usepackage[latin1]{inputenc}
\usepackage[T1]{fontenc}
\usepackage[francais]{babel}

\begin{document}
\renewcommand{\contentsname}{Sommaire}
\tableofcontents
\part{Configuration}
\chapter{Configuration g\'en\'erale}
Cet onglet vous permet de faire une configuration g\'en\'erale pour votre sc\`ene, on y retrouve les options principales : M\'ethode de rendu, antialiasing, rendu 3D, threads...
\section{M\'ethode de rendu}
\subsection{Rendu avec un seul thread}
Le rendu avec un seul thread rend accessible tout type de rendu :
  \begin{description}
    \item[Lin\'eaire horizontal :] mode de rendu horizontal classique
    \item[Lin\'eaire vertical :] mode de rendu vertical classique
    \item[D\'epix\'elisation :] mode de rendu permettant d'avoir tr\`es t\^ot une id\'ee du rendu final
    \item[Al\'eatoire horizontal / vertical :] m\^eme m\'ethode que les modes lin\'eaires sauf que les bandes sont choisies al\'eatoirement
    \item[Al\'eatoire :] mode de rendu totalement al\'eatoire pour chaque pixel
  \end{description}
\subsection{Rendu avec du multithreading}
Le rendu avec le multithreading ne pas permet d'utiliser de m\'ethode de rendu al\'eatoire.
\section{Threads}
Le multithreading permet de diminuer consid\'erablement le temps de rendu de vos images en utilisant tout les coeurs de votre processeur.
\newline Mettre un nombre de thread trop \'elev\'e est inutile (ex: 5 threads pour un process quad core).
\section{Antialiasing}
Augmenter l'antialiasing am\'eliore grandement la qualit\'e de l'image en recalculant plusieurs fois chaque pixel.
\newline ATTENTION: Cel\`a augmente consid\'erablement le temps de rendu !
\section{Kdtree}
Activer l'option kdtree permet d'augmenter la vitesse de rendu pour des sc\`enes compos\'es d'un grand nombre de triangles, cette option est inutile dans les autres cas.
\section{Profondeur de champ}
La profondeur de champ permet de rendre la sc\`ene nette en son centre et floue autour, ce qui donne un effet tr\'es r\'ealiste.
\newline ATTENTION: Cel\`a augmente consid\'erablement le temps de rendu !
\section{Rendu 3D}
Ce mode permet de visualiser votre sc\`enes en 3D \`a l'aide d'une paire de lunette 3D, un \'ecartement de 0.05 est souvent l'id\'eal.
\newline Il vaut mieux choisir des sc\`enes avec des objets ayant une position diff\'erente sur l'axe des abscisses les uns des autres.
\newline Cel\`a ralentit un peu le rendu car il faut calculer deux fois l'image.
\section{Autres options g\'en\'erale}
Dans ce panel vous pouvez aussi choisir de quelle cam\'era vous souhaitez voir la sc\`ene (les cam\'eras \'etant \`a d\'efinir dans le fichier .xml).
\newline Il est aussi possible de choisir la taille de l'image.
\chapter{Lumi\`eres}
L'onglet lumi\`ere propose des r\'eglages pour l'ensemble des lumi\`eres de la sc\`ene.
\section{Exposition}
Cette option permet de g\'erer l'exposition \`a la lumi\`ere, plus elle est faible plus la sc\`ene sera sombre.
\newline Utile pour assombrir une sc\`ene un peu trop sombre.
\section{Lumi\`ere directe}
Cette option permet de r\'egler l'intensit\'e de la lumi\`ere.
\section{Lumi\`ere diffuse}
La lumi\`ere diffuse permet de donner un effet plus r\'ealiste \`a la lumi\`ere en flouttant les zones \'eclair\'ees.
\newline ATTENTION: Cel\`a augmente consid\'erablement le temps de rendu.
\section{Lumi\`ere sp\'eculaire}
La lumi\`ere sp\'eculaire permet de d\'efinir la brillance, plus elle est \'elev\'ee et plus l'objet brille.
\section{Ombres diffuses}
Les ombres diffuses permettent de rendre les ombres plus r\'ealistes, le proc\'eder est le m\^eme que pour la lumi\`ere diffuse : floutter les zone d\'ombres.
\chapter{Mat\'eriaux}
L'onglet mat\'eriaux permet de r\'egler la r\'eflexion et la transparence.
\section{R\'eflexion}
Cette option permet de r\'egler la profondeur de la r\'eflexion, c\'est \`a dire combien de fois un objet peut r\'efl\'echir ce qui l'entoure (astuce: il est possible de faire r\'efl\'echir des objets entre eux en activant la r\'eflexion sur deux objets proches).
\newline La r\'eflexion diffuse permet de floutter la r\'eflexion.
\section{Transparence}
Cette option permet de g\'erer la profondeur de la transparence, plus elle est \'elev\'ee plus l'objet est transparent.
\chapter{Environnement}
Cet onglet permet de configurer une image de fond ou encore une lumi\`ere / couleur d'ambiance.
\section{Fonds}
\subsection{Cubemap}
Permet de charger une image en fond pour votre sc\`ene, \`a combiner avec la r\'eflexion.
\subsection{Couleur de fond}
Permet de choisir une couleur de fond pour votre sc\`ene.
\section{Lumi\`ere et couleur d'ambiance}
\subsection{Lumi\`ere d'ambiance}
Permet d'augmenter la couleur de la lumi\`ere.
\subsection{Couleur d'ambiance}
Permet d'ajouter une couleur g\'en\'erale \`a l'ensemble de votre sc\`ene.
\chapter{Illumination globale}
\section{Ambient Occlusion}
L'ambient occlusion est un proc\'ed\'e qui consiste \`a lancer un nombre de rayon \'egal \`a la valeur du sampling \`a partir du point d'intersection pour d\'efinir si le point est proche d'un objet ou non.
\newline Cela permet un effet tr\'es r\'ealiste sur la lumi\`ere mais une sc\`ene avec l'occlusion ambiente est tr\`es longue \`a charger.
\newline Un sampling de 50 minimum est conseill\'e.
\section{Photon mapping}
Cette option permet de g\'en\'erer une carte de photon qui permet d'obtenir un effet de lumi\`ere caustique et d'inter-re\'eflexion.
\part{Editeur de mat\'eriaux}
\chapter{Propri\'et\'es}
Le cadre propri\'et\'es vous permet de modifier des caracteristiques sur un type de mat\'eriel au lieu de les toucher tous d'un coup avec la configuration.
\newline On y trouve :
\begin{description}
 \item[Sp\'eculaire :] Permet de r\'egler la lumi\'ere sur cet objet
 \item[Puissance sp\'eculaire :] Permet d'augmenter l'intensit\'e de la lumi\`ere sur ce mat\'eriel.
 \item[R\'eflection :] Permet de d\'efinir le coefficient de r\'eflection pour le mat\'eriel, valeur comprise entre 0 et 1.
 \item[Transmission :] Permet de g\'erer la transparence. 
 \item[Indice de r\'efraction :] Permet de r\'egler la r\'efraction, valeur comprise entre 0 et 5.
 \item[R\'eflection diffuse :] Permet de r\'egler la r\'eflection diffuse afin de rendre la r\'eflection plus r\'ealiste
\end{description}
\chapter{Textures}
\section{Textures proc\'edurales}
Permet d'appliquer une texture d\'ej\`a d\'efinie par notre raytracer.
\newline Sont disponibles :
\begin{itemize}
 \item Damier
 \item Effet bois
 \item Bruit de perlin
 \item Effet marbre
\end{itemize}
\section{Textures personnelles}
Il est choisit possible de choisir sa propre texture, pour cel\`a il faut s\'electionner image comme type.
\newline Vous pouvez aussi personnaliser la r\'ep\'etition des textures suivant la taille de vos objets.
\chapter{D\'eformation}
\section{Bumpgmapping}
Le bumpmapping permet de d\'eformer la normale sur une texture.
\newline Il est possible d'appliquer un simple bruit de perlin sur la texture.
\newline Une utilisation plus avanc\'ee consiste \`a charger une heightmap pour d\'eformer la normale \`a partir de cette texture (faire la terre en relief sur une sph\`ere par exemple).
\newline On peut ainsi avoir une texture en relief en chargeant la m\^eme texture et heightmap.
\section{D\'eformation proc\'edurale}
La d\'eformation proc\'edurale permet de donner un effet de vague en modifiant l'une des composantes de la normales : abscisses, ordonn\'ees ou hauteurs.
\newline Le coefficient permet de d\'efinir la taille de la d\'eformation, plus il est \'elev\'e plus les vagues sont grosses.
\section{Texture limitante}
Choisir une texture limitante permet de couper un objet, en effet les parties transparentes de la texture correspondront \`a du vide sur l\'objet.
\part{Gestion du rendu}
\chapter{Charger et enregistrer une sc\`ene}
\section{Charger un fichier .xml}
Pour charger une sc\`ene au format .xml (qui peut contenir les donn\'ees d'un fichier 3ds max) utilisez le raccourci ctrl+O ou directement le menu fichier ou encore la toolbar.
Si votre fichier est invalide la console vous affichera un message d'erreur pr\'ecis afin d'\'editer ce dernier.
Voir partie III pour la cr\'eation de fichiers .xml.
\section{Enregistrer un rendu termin\'e}
Une fois votre rendu termin\'e vous pouvez l'enregistrer \`a l'aide du raccourci ctrl+S, ou encore du menu fichier ou de la toolbar.
Tout les formats d'images sont support\'es : png, bmp, jpeg, gif...
\section{Quitter et mettre la GUI en background}
Vous pouvez quitter la GUI avec le raccourci ctrl+Q, ou encore la mettre en background avec le raccourci ctrl+H.
Si vous mettez la GUI en background un message vous avertira quand le rendu sera termin\'e.
\chapter{G\'erer le d\'eroulement du rendu}
La gui dispose de 3 fonctions pour g\'erer le d\'eroulement de votre rendu :
\begin{description}
 \item [Lancer :] Cette commande \`a pour effet de d\'emarrer le rendu ou reprendre un rendu mis en pause.
 \item [Mettre en pause :] Cette commande \`a pour effet de mettre en pause le rendu, ce dernier peut \^etre repris par la suite
 \item [Stopper :] Cette commande stoppe le rendu, il n'est plus possible de reprendre ce dernier. A utiliser pour changer de sc\`ene ou reconfigurer.
\end{description}
\part{Fichiers .xml}
\chapter{Les mat\'eriaux}
Il est possible de cr\'eer des mat\'eriaux \`a l'aide d'un fichier xml, ils sont enti\`erement configurables \`a l'aide de diff\'erentes balises.
\section{Syntaxe}
Les mat\'eriaux doivent \^etre d\'eclar\'es entre les balises <materials></materials> \`a l'aide de la balise <material name="..."></material>
\newline Il est possible de d\'eclarer plusieurs mat\'eriaux \`a l'int\`erieur d'une balise <materials>.
\newline Les options du mat\'eriel se d\'eclareront de la mani\`ere <option>...</option>.
\section{Couleur}
On peut d\'efinir la couleur avec la balise <color>, elle peut \^etre indiqu\'e sous la forme R,G,B ou en hexadecimal.
\newline Exemple : <color>200,120,255<color> ou encore <color>FFFFFF</color>.
\section{Lumi\`ere sp\'eculaire}
On peut d\'efinir la lumi\`ere sp\'eculaire avec la balise <specular> (valeur entre 0 et 1), et sa puissance avec <specularPow>.
\section{R\'eflection et r\'efraction}
On peut d\'efinir la r\'eflection avec la balise <reflection> et une valeur comprise entre 0 et 1.
\newline Pour la r\'eflection diffuse on utilise <diffusedReflection> avec la valeur 1 (true).
\newline On peut aussi d\'efinir la profondeur de la r\'eflection avec <refractionIndex> suivis d'une valeur.
\newline Exemple :
\newline <specular>0.5</specular>\newline <specularPow>50</specularPow>\newline<reflection>1</reflection>\newline<refractionIndex>5</refractionIndex>\newline<diffusedReflection>1</diffusedReflection>
\chapter{Les objets}
On peut d\'efinir les objets entre les balises <objects> et </objects>, et dans la sous-balise <primitives>.
\newline Syntaxe : <objects>\newline<primitives>\newline</primitives>\newline</objects>.
\section{Les sph\`eres, les cylindres, les c\^ones et les cubes trou\'es}
Pour d\'efinir une sph\`ere on la d\'eclare avec la balise <primitive type="sphere" mat\'erial="...">.
\newline Une sph\`ere est d\'efinie par :
\begin{description}
 \item [Sa position :] Avec la balise <position> sous la forme <position>x,y,z</position>.
 \item [Sa rotation :] Avec la balise <rotation type="degree" ou "radian">x,y,z</rotation>.
 \item [Son rayon :] Avec la balise <radius>x</radius>.
\end{description}
Les cylindres et les c\^ones se d\'efinissent de la m\^eme mani\`ere en d\'eclarant type="cylinder" ou type="cone" au lieu de type="sph\`ere".
\newline Il en est de m\^eme pour les cubes trou\'es avec type="cubeTroue".
\section{Les plans}
Pour d\'efinir un plan on le d\'eclare avec la balise <primitive type="plane" mat\'erial="...">.
\newline Un plan est d\'efinie par :
\begin{description}
 \item [Sa position :] Avec la balise <position> sous la forme <position>x,y,z</position>.
 \item [Sa rotation :] Avec la balise <rotation type="degree" ou "radian">x,y,z</rotation>.
\end{description}
\section{Les tores}
Pour d\'efinir un tore on utilise la balise <primitive type="torus" mat\'erial="...">.
\newline Un tore est d\'efinie par :
\begin{description}
 \item [Sa position :] Avec la balise <position> sous la forme <position>x,y,z</position>.
 \item [Sa rotation :] Avec la balise <rotation type="degree" ou "radian">x,y,z</rotation>.
 \item [Son cercle mineur :] Avec la balise <minor>x</minor>.
 \item [Son cercle majeur :] Avec la balise <major>x</major>.
\end{description}
\section{Les parall\'el\'epip\`edes}
On peut d\'efinir un parall\'el\'epip\`ede sans balise <object> il suffit de d\'eclarer <parallelepipede mat\'erial="...">.
\newline Un parall\'el\'epip\`ede est d\'efinie par :
\begin{description}
 \item [Sa position :] Avec la balise <position> sous la forme <position>x,y,z</position>.
 \item [Sa rotation :] Avec la balise <rotation type="degree" ou "radian">x,y,z</rotation>.
 \item [Sa largeur :] Avec la balise <width>0,y,0</width>.
 \item [Sa profondeur :] Avec la balise <depth>x,0,0</depth>.
 \item [Sa hauteur :] Avec la balise <height>0,0,z</height>.
\end{description}
\chapter{Les lumi\`eres}
On peu d\'efinir des lumi\`eres avec un fichier .xml, apr\`es les objets, avec la balise <lights>.
\section{Les spots}
Pour d\'eclarer un spot on utilise la balise <light type="spot">.
\newline Un spot est d\'efinie par :
\begin{description}
 \item [Sa position :] Avec la balise <position>x,y,z</position>.
 \item [Sa couleur :] Avec la balise <color>R,G,B</color> ou <color>hexa</color>.
 \item [Sa lumi\`ere direct :] <directLight>3000</directLight>.
 \item [Son intensit\'e :] Avec la balise <intensity>x</intensity>.
\end{description}
\section{Les parall\'elogrammes}
Pour d\'eclarer une lumi\`ere avec un parallelogramme on utilise <light type="parallelogram">.
Un parallelogramme est d\'efinie par :
\begin{description}
 \item [Un point :] Avec la balise <point>x,y,z</point>.
 \item [Deux vecteurs :] Avec les balises <vector1>x,y,z</vector1> et <vector2>x,y,z</vector2>.
 \item [Sa couleur :] Avec la balise <color>R,G,B</color> ou <color>hexa</color>.
 \item [Sa lumi\`ere direct :] <directLight>3000</directLight>.
 \item [Son intensit\'e :] Avec la balise <intensity>x</intensity>.
\end{description}
\end{document}