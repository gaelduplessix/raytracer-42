\documentclass{report}

\usepackage[latin1]{inputenc}
\usepackage[T1]{fontenc}
\usepackage[francais]{babel}

\begin{document}
\renewcommand{\contentsname}{Sommaire}
\tableofcontents
\part{Configuration}
\chapter{Configuration g\'en\'erale}
Cet onglet vous permet de faire une configuration g\'en\'erale pour votre sc\`ene, on y retrouve les options principales : M\'ethode de rendu, antialiasing, rendu 3D, threads...
\section{M\'ethode de rendu}
\subsection{Rendu avec un seul thread}
Le rendu avec un seul thread rend accessible tout type de rendu :
  \begin{description}
    \item[Lin\'eaire horizontal :] mode de rendu horizontal classique
    \item[Lin\'eaire vertical :] mode de rendu vertical classique
    \item[D\'epix\'elisation :] mode de rendu permettant d'avoir tr\`es t\^ot une id\'ee du rendu final
    \item[Al\'eatoire horizontal / vertical :] m\^eme m\'ethode que les modes lin\'eaires sauf que les bandes sont choisies al\'eatoirement
    \item[Al\'eatoire :] mode de rendu totalement al\'eatoire pour chaque pixel
  \end{description}
\subsection{Rendu avec du multithreading}
Le rendu avec le multithreading ne pas permet d'utiliser de m\'ethode de rendu al\'eatoire.
\section{Threads}
Le multithreading permet de diminuer consid\'erablement le temps de rendu de vos images en utilisant tout les coeurs de votre processeur.
\newline Mettre un nombre de thread trop \'elev\'e est inutile (ex: 5 threads pour un process quad core).
\section{Antialiasing}
Augmenter l'antialiasing am\'eliore grandement la qualit\'e de l'image en recalculant plusieurs fois chaque pixel.
\newline ATTENTION: Cel\`a augmente consid\'erablement le temps de rendu !
\section{Kdtree}
Activer l'option kdtree permet d'augmenter la vitesse de rendu pour des sc\`enes compos\'es d'un grand nombre de triangles, cette option est inutile dans les autres cas.
\section{Profondeur de champ}
La profondeur de champ permet de rendre la sc\`ene nette en son centre et floue autour, ce qui donne un effet tr\'es r\'ealiste.
\newline ATTENTION: Cel\`a augmente consid\'erablement le temps de rendu !
\section{Rendu 3D}
Ce mode permet de visualiser votre sc\`enes en 3D \`a l'aide d'une paire de lunette 3D, un \'ecartement de 0.05 est souvent l'id\'eal.
\newline Il vaut mieux choisir des sc\`enes avec des objets ayant une position diff\'erente sur l'axe des abscisses les uns des autres.
\newline Cel\`a ralentit un peu le rendu car il faut calculer deux fois l'image.
\section{Autres options g\'en\'erale}
Dans ce panel vous pouvez aussi choisir de quelle cam\'era vous souhaitez voir la sc\`ene (les cam\'eras \'etant \`a d\'efinir dans le fichier .xml).
\newline Il est aussi possible de choisir la taille de l'image.
\chapter{Lumi\`eres}
L'onglet lumi\`ere propose des r\'eglages pour l'ensemble des lumi\`eres de la sc\`ene.
\section{Exposition}
Cette option permet de g\'erer l'exposition \`a la lumi\`ere, plus elle est faible plus la sc\`ene sera sombre.
\newline Utile pour assombrir une sc\`ene un peu trop sombre.
\section{Lumi\`ere directe}
Cette option permet de r\'egler l'intensit\'e de la lumi\`ere.
\section{Lumi\`ere diffuse}
La lumi\`ere diffuse permet de donner un effet plus r\'ealiste \`a la lumi\`ere en flouttant les zones \'eclair\'ees.
\newline ATTENTION: Cel\`a augmente consid\'erablement le temps de rendu.
\section{Lumi\`ere sp\'eculaire}
La lumi\`ere sp\'eculaire permet de d\'efinir la brillance, plus elle est \'elev\'ee et plus l'objet brille.
\section{Ombres diffuses}
Les ombres diffuses permettent de rendre les ombres plus r\'ealistes, le proc\'eder est le m\^eme que pour la lumi\`ere diffuse : floutter les zone d\'ombres.
\chapter{Mat\'eriaux}
L'onglet mat\'eriaux permet de r\'egler la r\'eflexion et la transparence.
\section{R\'eflexion}
Cette option permet de r\'egler la profondeur de la r\'eflexion, c\'est \`a dire combien de fois un objet peut r\'efl\'echir ce qui l'entoure (astuce: il est possible de faire r\'efl\'echir des objets entre eux en activant la r\'eflexion sur deux objets proches).
\newline La r\'eflexion diffuse permet de floutter la r\'eflexion.
\section{Transparence}
Cette option permet de g\'erer la profondeur de la transparence, plus elle est \'elev\'ee plus l'objet est transparent.
\chapter{Environnement}
Cet onglet permet de configurer une image de fond ou encore une lumi\`ere / couleur d'ambiance.
\section{Fonds}
\subsection{Cubemap}
Permet de charger une image en fond pour votre sc\`ene, \`a combiner avec la r\'eflexion.
\subsection{Couleur de fond}
Permet de choisir une couleur de fond pour votre sc\`ene.
\section{Lumi\`ere et couleur d'ambiance}
\subsection{Lumi\`ere d'ambiance}
Permet d'augmenter la couleur de la lumi\`ere.
\subsection{Couleur d'ambiance}
Permet d'ajouter une couleur g\'en\'erale \`a l'ensemble de votre sc\`ene.
\chapter{Illumination globale}
\section{Ambient Occlusion}
L'ambient occlusion est un proc\'ed\'e qui consiste \`a lancer un nombre de rayon \'egal \`a la valeur du sampling \`a partir du point d'intersection pour d\'efinir si le point est proche d'un objet ou non.
\newline Cela permet un effet tr\'es r\'ealiste sur la lumi\`ere mais une sc\`ene avec l'occlusion ambiente est tr\`es longue \`a charger.
\newline Un sampling de 50 minimum est conseill\'e.
\section{Photon mapping}
Cette option permet de g\'en\'erer une carte de photon qui permet d'obtenir un effet de lumi\`ere caustique et d'inter-re\'eflexion.
\end{document}